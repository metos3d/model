%
% Metos3D: A Marine Ecosystem Toolkit for Optimization and Simulation in 3-D
% Copyright (C) 2012  Jaroslaw Piwonski, CAU, jpi@informatik.uni-kiel.de
%
% This program is free software: you can redistribute it and/or modify
% it under the terms of the GNU General Public License as published by
% the Free Software Foundation, either version 3 of the License, or
% (at your option) any later version.
%
% This program is distributed in the hope that it will be useful,
% but WITHOUT ANY WARRANTY; without even the implied warranty of
% MERCHANTABILITY or FITNESS FOR A PARTICULAR PURPOSE.  See the
% GNU General Public License for more details.
%
% You should have received a copy of the GNU General Public License
% along with this program.  If not, see <http://www.gnu.org/licenses/>.
%

%
%	document class
%
\documentclass{article}

%
%	packages
%
\usepackage{listings}
\usepackage{natbib}
\usepackage{amsmath}
\usepackage[pdfborder={0 0 0},colorlinks,urlcolor=blue]{hyperref}

%
%	BEGIN DOCUMENT
%
\begin{document}
%
%	title
%
\title{
Metos3D \\
\medskip
\texttt{model}
}
\author{
Jaroslaw Piwonski\thanks{\texttt{jpi@informatik.uni-kiel.de}} \,,
Thomas Slawig\thanks{\texttt{ts@informatik.uni-kiel.de},
both: Department of Computer Science, Algorithmic Optimal Control -- Computational Marine Science,
Excellence Cluster The Future Ocean, Christian-Albrechts-Platz 4, 24118 Kiel, Germany.}
}
\date{\today}
\maketitle

%
%	Model interface
%
\section{Model interface}

Metos3D \citep[][]{PiwSla16} can be coupled to every (biogeochemical) model that
conforms to the following interface:

\begin{verbatim}
subroutine metos3dbgc(ny, nx, nu, nb, nd, dt, q, t, y, u, b, d, ndiag, diag)
    integer :: ny              ! tracer count
    integer :: nx              ! layer count
    integer :: nu              ! parameter count
    integer :: nb              ! boundary condition count
    integer :: nd              ! domain condition count
    integer :: ndiag           ! diagnostic variable count
    real*8  :: dt              ! ocean time step
    real*8  :: q(nx, ny)       ! bgc model output
    real*8  :: t               ! point in time
    real*8  :: y(nx, ny)       ! bgc model input
    real*8  :: u(nu)           ! parameters
    real*8  :: b(nb)           ! boundary conditions
    real*8  :: d(nx, nd)       ! domain conditions
    real*8  :: diag(nx, ndiag) ! diagnostic variables
end subroutine
\end{verbatim}

The interface decouples biogeochemical models and driver routines
(ocean circulation, forcing, geometry) programmatically.
%
It gives you the possibility to provide a free number of tracers,
parameters, boundary and domain conditions. It suits well an
optimization as well as an Automatic Differentiation (AD) context.

The interface changed slightly since it was introduced for the first time.
The initial version can be found at \citep[][]{PiwSla16}.

%
%
%	References	%%%%%%%%%%%%%%%%%%%%%%%%%%%%%%%%%%%%%%%%%%%%%%%%
%
\bibliographystyle{plain}
\bibliography{/Users/jpicau/Documents/work/svn/Literature/literature}
%\begin{thebibliography}{1}

\bibitem{DuSoScSt05}
Stephanie Dutkiewicz, Andrei~P. Sokolov, Jeffery Scott, and Peter~H. Stone.
\newblock {A} three-dimensional ocean-seaice-carbon cycle model and its
  coupling to a two-dimensional atmospheric model: {U}ses in climate change
  studies.
\newblock Technical Report 122, MIT Joint Program on the Science and Policy of
  Global Change, 2005.

\bibitem{MAHPH97}
J.~Marshall, A.~Adcroft, C.~Hill, L.~Perelman, and C.~Heisey.
\newblock A finite-volume, incompressible navier stokes model for studies of
  the ocean on parallel computers.
\newblock {\em Journal of Geophysical Research}, 102:5753--5766, 1997.

\end{thebibliography}


\end{document}


