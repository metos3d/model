%
% Metos3D: A Marine Ecosystem Toolkit for Optimization and Simulation in 3-D
% Copyright (C) 2012  Jaroslaw Piwonski, CAU, jpi@informatik.uni-kiel.de
%
% This program is free software: you can redistribute it and/or modify
% it under the terms of the GNU General Public License as published by
% the Free Software Foundation, either version 3 of the License, or
% (at your option) any later version.
%
% This program is distributed in the hope that it will be useful,
% but WITHOUT ANY WARRANTY; without even the implied warranty of
% MERCHANTABILITY or FITNESS FOR A PARTICULAR PURPOSE.  See the
% GNU General Public License for more details.
%
% You should have received a copy of the GNU General Public License
% along with this program.  If not, see <http://www.gnu.org/licenses/>.
%

%
%	document class
%
\documentclass{article}
%
%	packages
%
\usepackage{natbib}
\usepackage{amsmath}
\usepackage[pdfborder={0 0 0},colorlinks,urlcolor=blue]{hyperref}
%
%	new commands
%
\newcommand{\lmb}{{\lambda}}
\newcommand{\sig}{\sigma}
\newcommand{\sigb}{\bar{\sigma}}
\newcommand{\Om}{\Omega}
\newcommand{\dd}{\partial}
%\newcommand{\kap}{\kappa}
%
%	commands
%
\parindent0em
%
%	BEGIN DOCUMENT
%
\begin{document}
%
%	title
%
\title{
Metos3D \\
\medskip
\texttt{model}
}
\author{
Jaroslaw Piwonski\thanks{\texttt{jpi@informatik.uni-kiel.de}} \,,
Thomas Slawig\thanks{\texttt{ts@informatik.uni-kiel.de},
both: Department of Computer Science, Algorithmic Optimal Control -- Computational Marine Science,
Excellence Cluster The Future Ocean, Christian-Albrechts-Platz 4, 24118 Kiel, Germany.}
}
\date{\today}
\maketitle

%
%	Model interface
%
\section{Model interface}

Metos3D can be coupled to every (biogeochemical) model that conforms to the following interface:

\begin{verbatim}
subroutine metos3dbgc(n, ny, m, nb, nd, dt, q, t, y, u, b, d)
    integer :: n           ! tracer count
    integer :: ny          ! layer count
    integer :: m           ! parameter count
    integer :: nb          ! boundary condition count
    integer :: nd          ! domain condition count
    real*8  :: dt          ! ocean time step
    real*8  :: q(nz, n)    ! bgc model output
    real*8  :: t           ! point in time
    real*8  :: y(nz, n)    ! bgc model input
    real*8  :: u(m)        ! parameters
    real*8  :: b(nb)       ! boundary conditions
    real*8  :: d(nz, nd)   ! domain conditions
end subroutine
\end{verbatim}

The interface decouples biogeochemical models and driver routines
(ocean circulation, forcing, geometry) programmatically.
%
It gives you the possibility to provide a free number of tracers,
parameters, boundary and domain conditions. It suits well an
optimization as well as an Automatic Differentiation (AD) context.
%


%
%	BGC Models
%
\section{BGC Models}

Every model archive contains an \texttt{option} directory.
%
You can find a test option file therein.
%
Use it as a starting point for your own work.

%
%	I-Cs
%
\subsection{\texttt{I-Cs}}

The Iodine (I) and Caesium (Cs) model was implemented to the predict
the Caesium distribution after the Fukushima accident.

%
%	Equations
%
\subsubsection{Equations}

The model equations describe the radioactive decay of the $I^{131}$ and $Cs^{137}$
isotops named $y_1$ and $y_2$, respectively. The decay depends on the
half-life. The tracers do not react with each other. The equations read:
\begin{align*}
q_1(y_1, y_2)	& = \log(0.5) \, 360.0 / 8.02070 \, y_1 \\
q_2(y_1, y_2)	& = \log(0.5) \, 1.0 / 30.17 \, y_2
\end{align*}

%
%	MITgcm-PO4-DOP
%
\subsection{\texttt{MITgcm-PO4-DOP}}

The \texttt{MITgcm-PO4-DOP} model is an \emph{original} implementation of
a biogeochemical model that is used for the 
MIT General Circulation Model \citep[cf.][MITgcm]{MAHPH97}
biogeochemistry tutorial and described in detail in
\citep[][]{DuSoScSt05}.
%
The model comprises five biogeochemical variables,
namely
dissolved inorganic carbon (DIC),
alkalinity (ALK),
phosphate (PO4),
dissolved organic phosphorous (DOP) and
oxygen (O2).
%
In fact,
just PO4 and DOP are used here since
the concentrations of DIC, ALK and O2 are derived from those two.

%
%
%	References	%%%%%%%%%%%%%%%%%%%%%%%%%%%%%%%%%%%%%%%%%%%%%%%%
%
\bibliographystyle{plain}
\bibliography{/Users/jpicau/Documents/ARBEIT/CODE/Literature/literature}
%\begin{thebibliography}{1}

\bibitem{DuSoScSt05}
Stephanie Dutkiewicz, Andrei~P. Sokolov, Jeffery Scott, and Peter~H. Stone.
\newblock {A} three-dimensional ocean-seaice-carbon cycle model and its
  coupling to a two-dimensional atmospheric model: {U}ses in climate change
  studies.
\newblock Technical Report 122, MIT Joint Program on the Science and Policy of
  Global Change, 2005.

\bibitem{MAHPH97}
J.~Marshall, A.~Adcroft, C.~Hill, L.~Perelman, and C.~Heisey.
\newblock A finite-volume, incompressible navier stokes model for studies of
  the ocean on parallel computers.
\newblock {\em Journal of Geophysical Research}, 102:5753--5766, 1997.

\end{thebibliography}


\end{document}


