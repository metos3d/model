%
% Metos3D: A Marine Ecosystem Toolkit for Optimization and Simulation in 3-D
% Copyright (C) 2012  Jaroslaw Piwonski, CAU, jpi@informatik.uni-kiel.de
%
% This program is free software: you can redistribute it and/or modify
% it under the terms of the GNU General Public License as published by
% the Free Software Foundation, either version 3 of the License, or
% (at your option) any later version.
%
% This program is distributed in the hope that it will be useful,
% but WITHOUT ANY WARRANTY; without even the implied warranty of
% MERCHANTABILITY or FITNESS FOR A PARTICULAR PURPOSE.  See the
% GNU General Public License for more details.
%
% You should have received a copy of the GNU General Public License
% along with this program.  If not, see <http://www.gnu.org/licenses/>.
%
%   metos3d-model.tex

%
%	document class
%
\documentclass{article}
%
%	packages
%
\usepackage{amsmath}
\usepackage[pdfborder={0 0 0},colorlinks,urlcolor=blue]{hyperref}
%
%	new commands
%
\newcommand{\lmb}{{\lambda}}
\newcommand{\sig}{\sigma}
\newcommand{\sigb}{\bar{\sigma}}
\newcommand{\Om}{\Omega}
\newcommand{\dd}{\partial}
%\newcommand{\kap}{\kappa}
%
%	commands
%
\parindent0em
%
%	BEGIN DOCUMENT
%
\begin{document}
%
%	title
%
\title{
Metos3D \\
\bigskip
Biogeochemical Models
}
\author{
Jaroslaw Piwonski\thanks{\texttt{jpi@informatik.uni-kiel.de}} \,,
Thomas Slawig\thanks{\texttt{ts@informatik.uni-kiel.de},
both: Department of Computer Science, Algorithmic Optimal Control -- Computational Marine Science,
Excellence Cluster The Future Ocean, Christian-Albrechts-Platz 4, 24118 Kiel, Germany.}
}
\date{\today}
\maketitle

%
%	Introdution
%
\section{Introdution}

Metos3D can be coupled to every (biogeochemical) model that conforms to the following interface:

\begin{verbatim}
subroutine metos3dbgc(n, nz, m, nb, nd, dt, q, t, y, u, b, d)
    integer :: n           ! tracer count
    integer :: nz          ! layer count
    integer :: m           ! parameter count
    integer :: nb          ! boundary condition count
    integer :: nd          ! domain condition count
    real*8  :: dt          ! ocean time step
    real*8  :: q(nz, n)    ! bgc model output
    real*8  :: t           ! point in time
    real*8  :: y(nz, n)    ! bgc model input
    real*8  :: u(m)        ! parameters
    real*8  :: b(nb)       ! boundary conditions
    real*8  :: d(nz, nd)   ! domain conditions
end subroutine
\end{verbatim}

The interface decouples biogeochemical models and driver routines (ocean circulation, forcing, geometry) programmatically. It gives you the possibility to provide a free number of tracers, parameters, boundary and domain conditions. It suits well an optimization as well as an Automatic Differentiation (AD) context.

%
%	BGC Models
%
\section{BGC Models}

Every model archive contains an \texttt{option} and a \texttt{job}
directory. You can find examples of option files in the former
and job files for high performance clusters in the latter.
Use them as a starting point for your own work.

%
%	I-Cs
%
\subsection{I-Cs}

The Iodine (I) and Caesium (Cs) model was actually implemented to show the flexibility of Metos3D.
It was used to the predict the Caesium distribution after the Fukushima accident.
See \href{http://www.ozean-der-zukunft.de/fukushima/}{http://www.ozean-der-zukunft.de/fukushima/}
for more (german only).

%
%	Equations
%
\subsubsection{Equations}

The model equations describe the radioactive decay of the $I^{131}$ and $Cs^{137}$
isotops named $y_1$ and $y_2$, respectively. The decay depends on the
half-life. The tracers do not react with each other. The equations read:
\begin{align*}
q_1(y_1, y_2)	& = \log(0.5) \, 360.0 / 8.02070 \, y_1 \\
q_2(y_1, y_2)	& = \log(0.5) \, 1.0 / 30.17 \, y_2
\end{align*}

%
%	N-DOP
%
\subsection{N-DOP}

The nutrients (N) and dissolved organic phosphorous (DOP) model is a simple and
well known one among biogeochemical modelers. It is based on the model presented
in \cite{PaFoBo05}.

%
%	Equations
%
\subsubsection{Equations}

The tracers are denoted by $ y = (y_1, y_2)^T = (y_{PO4}, y_{DOP})^T$.
The biological production (the net community productivity) is calculated as a function
$f_1$ of nutrients and light $I$. The production is limited using a 
half saturation function, also known as Michaelis-Menten kinetics,
and a maximum production rate parameter $\alpha$.
%
\begin{align*}
f_1(y_1,I)	& = \alpha \, \frac{y_1}{y_1 + K_1} \, \frac{I}{I+K_I}
\end{align*}
%
Light, here, is a portion of short wave radiation $I_{SWR}$, which is computed as a
function of latitude and season following the astronomical formula of Paltridge and Platt \cite{PalPla76}.
The portion depends on the photo-synthetically available radiation $\sig_{PAR}$,
the ice cover $\sig_{ice}$ and the exponential attenuation of water.
%
\begin{align*}
I		& = I_{SWR} \, \sig_{PAR} \, (1 - \sig_{ice}) \, \exp( - z \, K_{H2O} )
\end{align*}
%
A fraction of the biological production $\sig_2$ remains suspended in the water column
as dissolved organic phosphorus, which remineralizes with a rate $\lmb'_2$.
The remainder of the production sinks as particulate to depth where it is
remineralized according to the empirical power law relationship determined by
Martin et al. \cite{MaKnKaBr87}. Similar descriptions for biological production can be
found in \cite{PaFoBo05}, \cite{DuFoPa05} and \cite{YamTaj97}.

Moreover the model formulation consists of a production (sun lit, euphotic) zone,
with a depth of $l'$, and a
noneuphotic zone, $\Om_1$ and $\Om_2$ respectively.
The equations read
%
\begin{align*}
q_1(y) & = - f_1(y_1,I)						+ \lmb'_2 \, y_2	& \text{in $\Om_1$} \\
q_1(y) & = + \sigb_2 \, \dd_z \, F_1(y_1,I)	+ \lmb'_2 \, y_2	& \text{in $\Om_2$} \\
 & \\
q_2(y) & = + \sig_2 \, f_1(y_1,I)			- \lmb'_2 \, y_2	& \text{in $\Om_1$} \\
q_2(y) & =									- \lmb'_2 \, y_2	& \text{in $\Om_2$,}
\end{align*}
%
where
%
\begin{align*}
F_1(y_1,I)	& = (z/l')^{-b} \int_0^{l'} \! f_1(y_1,I) \, d\xi \,.
\end{align*}
%
The following table summarizes the parameters within the N-DOP model:
\begin{center}
\begin{tabular}{c|l|c}
Symbol		& Description								& Unit \\ \hline
$\alpha$		& maximum community production rate			& $ 1 / y $ \\
$K_{H2O}$	& attenuation of water						& $ 1 / m $ \\
$K_1$		& half saturation constant of PO4 			& $ m \, mol P / m^3 $ \\
$K_I$		& half saturation constant of light	 		& $ W / m^2 $ \\
$\lmb'_2$	& remineralization rate of DOP 				& $1 / d$ \\
$\sig_2$ 	& fraction of DOP, $\sigb_2 = (1-\sig_2)$ 	& $ - $ \\
$b$			& sinking velocity exponent					& $ - $
\end{tabular}
\end{center}

%
%	References	%%%%%%%%%%%%%%%%%%%%%%%%%%%%%%%%%%%%%%%%%%%%%%%%
%
\bibliographystyle{plain}
%\bibliography{/Users/jpicau/Documents/ARBEIT/CODE/Literature/literature}
\begin{thebibliography}{1}

\bibitem{DuFoPa05}
S.~Dutkiewicz, M.~Follows, and P.~Parekh.
\newblock Interactions of the iron and phosphorus cycles: A three-dimensional
  model study.
\newblock {\em Global Biogeochemical Cycles}, 19:1--22, 2005.

\bibitem{MaKnKaBr87}
J.~H. Martin, G.~A. Knauer, D.~M. Karl, and W.~W. Broenkow.
\newblock Vertex: carbon cycling in the northeast pacific.
\newblock {\em Deep Sea Research Part A. Oceanographic Research Papers},
  34(2):267--285, 1987.

\bibitem{PalPla76}
G.~W. Paltridge and C.~M.~R. Platt.
\newblock Radiative processes in meteorology and climatology, 318 pp, 1976.

\bibitem{PaFoBo05}
P.~Parekh, M.~J. Follows, and E.~A. Boyle.
\newblock Decoupling iron and phosphate in the global ocean.
\newblock {\em Global Biogeochemical Cycles}, 19, 2005.

\bibitem{YamTaj97}
Y.~Yamanaka and E.~Tajika.
\newblock Role of dissolved organic matter in the marine biogeochemical cycle:
  Studies using an ocean biogeochemical general circulation model.
\newblock {\em Global Biogeochemical Cycles}, 11(4):599--612, 1997.

\end{thebibliography}


\end{document}
